\section{N:/Documents/RTOS - CSC460/Assignments/Assignment2/Assignment2C/OS/crt0.c File Reference}
\label{crt0_8c}\index{N:/Documents/RTOS - CSC460/Assignments/Assignment2/Assignment2C/OS/crt0.c@{N:/Documents/RTOS - CSC460/Assignments/Assignment2/Assignment2C/OS/crt0.c}}
A C runtime. 

{\tt \#include $<$avr/io.h$>$}\par
{\tt \#include $<$avr/sfr\_\-defs.h$>$}\par
{\tt \#include \char`\"{}os.h\char`\"{}}\par
\subsection*{Defines}
\begin{CompactItemize}
\item 
\#define {\bf zero\_\-reg}~\char`\"{}r1\char`\"{}
\item 
\#define {\bf vector}(name)
\begin{CompactList}\small\item\em A macro to simplify the vector list. \item\end{CompactList}\end{CompactItemize}
\subsection*{Functions}
\begin{CompactItemize}
\item 
void {\bf \_\-\_\-vectors} (void)
\begin{CompactList}\small\item\em The vectors section. \item\end{CompactList}\item 
void {\bf \_\-\_\-vector\_\-not\_\-set} (void)
\begin{CompactList}\small\item\em A default routine that is called when an interrupt occurs for which no ISR was assigned. \item\end{CompactList}\item 
void {\bf \_\-\_\-init} (void)
\begin{CompactList}\small\item\em The beginning of the executable code in this file. \item\end{CompactList}\item 
void {\bf init2} (void)
\begin{CompactList}\small\item\em init2 \item\end{CompactList}\item 
void {\bf \_\-\_\-do\_\-copy\_\-data} (void)
\begin{CompactList}\small\item\em init4 \item\end{CompactList}\item 
void {\bf init9} (void)
\begin{CompactList}\small\item\em init9 \item\end{CompactList}\end{CompactItemize}


\subsection{Detailed Description}
A C runtime. 

Use only one of crt0.S and {\bf crt0.c}{\rm (p.\,\pageref{crt0_8c})}

This file is adapted from grct1.S in the avr source.

For use in Mantis Cheng's CSC 460 Fall 2007

To use this as startup code in AVR Studio, add \char`\"{}-nostartfiles\char`\"{} to [Linker Options] in Project$>$Configuration options$>$Custom Options.

The name of the function in the last call is where the program starts. (For project 2 it should be \char`\"{}OS\_\-Init\char`\"{}.)

\begin{Desc}
\item[Author:]Scott Craig 

Justin Tanner \end{Desc}


\subsection{Define Documentation}
\index{crt0.c@{crt0.c}!vector@{vector}}
\index{vector@{vector}!crt0.c@{crt0.c}}
\subsubsection{\setlength{\rightskip}{0pt plus 5cm}\#define vector(name)}\label{crt0_8c_cb2b81d3f4ce3e547f74901632950a26}


{\bf Value:}

\footnotesize\begin{verbatim}asm(\
    ".weak "  name "\n\t"\
    ".set  "  name " , __vector_not_set\n\t"\
    "jmp   "  name "\n\t"::);
\end{verbatim}\normalsize 
A macro to simplify the vector list. 

The symbol \char`\"{}\_\-\_\-vector\_\-i\char`\"{} is weakly bound to this spot in the object file. Later, other object files can reference this spot using this symbol.

The value of the symbol is set to \char`\"{}\_\-\_\-vector\_\-not\_\-set\char`\"{}, which is the label of a function below. Other files will change this if an ISR is declared.

The instruction at this spot is \char`\"{}jmp (addr)\char`\"{} (4 bytes). These addresses are hardwired in the mcu. \index{crt0.c@{crt0.c}!zero_reg@{zero\_\-reg}}
\index{zero_reg@{zero\_\-reg}!crt0.c@{crt0.c}}
\subsubsection{\setlength{\rightskip}{0pt plus 5cm}\#define zero\_\-reg~\char`\"{}r1\char`\"{}}\label{crt0_8c_24246e16d2ca770eab40983995d70ac1}


The \char`\"{}zero\char`\"{} register 

\subsection{Function Documentation}
\index{crt0.c@{crt0.c}!__do_copy_data@{\_\-\_\-do\_\-copy\_\-data}}
\index{__do_copy_data@{\_\-\_\-do\_\-copy\_\-data}!crt0.c@{crt0.c}}
\subsubsection{\setlength{\rightskip}{0pt plus 5cm}void \_\-\_\-do\_\-copy\_\-data (void)}\label{crt0_8c_ab2f2df3d3b1d20c55234f883bbcb9a3}


init4 

Copy data from \_\-\_\-data\_\-load\_\-start in program memory to \_\-\_\-data\_\-start in SRAM, initializing data in the process. A similar routine with the same name is defined in libgcc.S. This routine overrides it. \index{crt0.c@{crt0.c}!__init@{\_\-\_\-init}}
\index{__init@{\_\-\_\-init}!crt0.c@{crt0.c}}
\subsubsection{\setlength{\rightskip}{0pt plus 5cm}void \_\-\_\-init (void)}\label{crt0_8c_c3d17760e0044b1a3768a6fc7fd0bec8}


The beginning of the executable code in this file. 

The section names tell the linker where to place the code as specified in the linker script. eg. avr5.x \index{crt0.c@{crt0.c}!__vector_not_set@{\_\-\_\-vector\_\-not\_\-set}}
\index{__vector_not_set@{\_\-\_\-vector\_\-not\_\-set}!crt0.c@{crt0.c}}
\subsubsection{\setlength{\rightskip}{0pt plus 5cm}\_\-\_\-vector\_\-not\_\-set (void)}\label{crt0_8c_b0ab8d7ade49fa69ba562f88ded702a4}


A default routine that is called when an interrupt occurs for which no ISR was assigned. 

The default action is to reset, but it could be changed to do something else. \index{crt0.c@{crt0.c}!__vectors@{\_\-\_\-vectors}}
\index{__vectors@{\_\-\_\-vectors}!crt0.c@{crt0.c}}
\subsubsection{\setlength{\rightskip}{0pt plus 5cm}void \_\-\_\-vectors (void)}\label{crt0_8c_2b11cbc560dbd3b845b049fbd3ceefe3}


The vectors section. 

The numbers are off by 1 from the hardware manual, but consistent with iousbxx6\_\-7.h. Vector \char`\"{}0\char`\"{} is the reset vector, which jumps to the executable code.

Any interrupt ISR definition in the C code will overwrite these default definitions. \index{crt0.c@{crt0.c}!init2@{init2}}
\index{init2@{init2}!crt0.c@{crt0.c}}
\subsubsection{\setlength{\rightskip}{0pt plus 5cm}void init2 (void)}\label{crt0_8c_a4061ed2b7f0bd39d5b6dfb89feab723}


init2 

Clear the \char`\"{}zero\char`\"{} register, clear the status register, and set the stack pointer. \index{crt0.c@{crt0.c}!init9@{init9}}
\index{init9@{init9}!crt0.c@{crt0.c}}
\subsubsection{\setlength{\rightskip}{0pt plus 5cm}void init9 (void)}\label{crt0_8c_16bccc8605ebdf78fd0649215c467679}


init9 

The last of the init functions. Usually this would be the jump to \char`\"{}main()\char`\"{}

exit is defined in libgcc.S. It is an rjmp to itself. If the function called in init9 returns, it returns here. 