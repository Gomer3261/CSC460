\section{N:/Documents/RTOS - CSC460/Assignments/Assignment2/Assignment2C/OS/os.h File Reference}
\label{os_8h}\index{N:/Documents/RTOS - CSC460/Assignments/Assignment2/Assignment2C/OS/os.h@{N:/Documents/RTOS - CSC460/Assignments/Assignment2/Assignment2C/OS/os.h}}
\subsection*{Defines}
\begin{CompactItemize}
\item 
\#define {\bf MAXPROCESS}~8
\item 
\#define {\bf MAXEVENT}~8
\item 
\#define {\bf WORKSPACE}~512
\item 
\#define {\bf TICK}~5
\item 
\#define {\bf SYSTEM}~3
\item 
\#define {\bf PERIODIC}~2
\item 
\#define {\bf RR}~1
\item 
\#define {\bf NULL}~0
\item 
\#define {\bf IDLE}~0
\end{CompactItemize}
\subsection*{Typedefs}
\begin{CompactItemize}
\item 
typedef event {\bf EVENT}
\end{CompactItemize}
\subsection*{Functions}
\begin{CompactItemize}
\item 
void {\bf OS\_\-Init} (void)
\begin{CompactList}\small\item\em Setup the RTOS and create {\bf main()}{\rm (p.\,\pageref{_assignment2_8c_840291bc02cba5474a4cb46a9b9566fe})} as the first SYSTEM level task. \item\end{CompactList}\item 
void {\bf OS\_\-Abort} (void)
\begin{CompactList}\small\item\em Abort the execution of this RTOS due to an unrecoverable erorr. \item\end{CompactList}\item 
int {\bf Task\_\-Create} (void($\ast$f)(void), int arg, unsigned int level, unsigned int name)
\item 
void {\bf Task\_\-Terminate} (void)
\begin{CompactList}\small\item\em The calling task terminates itself. \item\end{CompactList}\item 
void {\bf Task\_\-Next} (void)
\begin{CompactList}\small\item\em The calling task gives up its share of the processor voluntarily. \item\end{CompactList}\item 
int {\bf Task\_\-Get\-Arg} (void)
\begin{CompactList}\small\item\em Retrieve the assigned parameter. \item\end{CompactList}\item 
{\bf EVENT} $\ast$ {\bf Event\_\-Init} (void)
\begin{CompactList}\small\item\em Initialize a new, non-NULL Event descriptor. \item\end{CompactList}\item 
void {\bf Event\_\-Wait} ({\bf EVENT} $\ast$e)
\begin{CompactList}\small\item\em Wait for the next occurrence of a signal on {\em e\/}. The calling process always blocks. \item\end{CompactList}\item 
void {\bf Event\_\-Signal} ({\bf EVENT} $\ast$e)
\begin{CompactList}\small\item\em Resume a {\bf single} waiting task on {\em e\/}. It is a {\em no-op\/} if there is no waiting process. \item\end{CompactList}\item 
void {\bf Event\_\-Broadcast} ({\bf EVENT} $\ast$e)
\begin{CompactList}\small\item\em Resume {\bf ALL} waiting tasks on {\em e\/}. It is a {\em no-op\/} if there is no waiting process. \item\end{CompactList}\item 
void {\bf Signal\_\-And\_\-Next} ({\bf EVENT} $\ast$e)
\begin{CompactList}\small\item\em Resume a waiting task on {\em e\/} and at the same time relinquish the processor. \item\end{CompactList}\item 
void {\bf Broadcast\_\-And\_\-Next} ({\bf EVENT} $\ast$e)
\begin{CompactList}\small\item\em Resume {\bf ALL} waiting tasks on {\em e\/} and at the same time relinquish the processor. \item\end{CompactList}\end{CompactItemize}
\subsection*{Variables}
\begin{CompactItemize}
\item 
const unsigned char {\bf PPP} [$\,$]
\item 
const unsigned int {\bf PT}
\end{CompactItemize}


\subsection{Define Documentation}
\index{os.h@{os.h}!IDLE@{IDLE}}
\index{IDLE@{IDLE}!os.h@{os.h}}
\subsubsection{\setlength{\rightskip}{0pt plus 5cm}\#define IDLE~0}\label{os_8h_9c21a7caee326d7803b94ae1952b27ca}


\index{os.h@{os.h}!MAXEVENT@{MAXEVENT}}
\index{MAXEVENT@{MAXEVENT}!os.h@{os.h}}
\subsubsection{\setlength{\rightskip}{0pt plus 5cm}\#define MAXEVENT~8}\label{os_8h_2a50d99dc5d43eb57bdcb2c4c29bc76e}


max. number of Events supported \index{os.h@{os.h}!MAXPROCESS@{MAXPROCESS}}
\index{MAXPROCESS@{MAXPROCESS}!os.h@{os.h}}
\subsubsection{\setlength{\rightskip}{0pt plus 5cm}\#define MAXPROCESS~8}\label{os_8h_c31d2fa7fae458de0b2d85a573fdd578}


max. number of processes supported \index{os.h@{os.h}!NULL@{NULL}}
\index{NULL@{NULL}!os.h@{os.h}}
\subsubsection{\setlength{\rightskip}{0pt plus 5cm}\#define NULL~0}\label{os_8h_070d2ce7b6bb7e5c05602aa8c308d0c4}


\index{os.h@{os.h}!PERIODIC@{PERIODIC}}
\index{PERIODIC@{PERIODIC}!os.h@{os.h}}
\subsubsection{\setlength{\rightskip}{0pt plus 5cm}\#define PERIODIC~2}\label{os_8h_f36821ad7b93ab31dcfaaa25e134fdf0}


a scheduling level: periodic tasks with predefined intervals \begin{Desc}
\item[See also:]{\bf PERIODIC TASKS}{\rm (p.\,\pageref{index_periodic})}, {\bf Task\_\-Create()}{\rm (p.\,\pageref{os_8c_d26f7e7c3185a703e89e97547747b03e})}. \end{Desc}
\index{os.h@{os.h}!RR@{RR}}
\index{RR@{RR}!os.h@{os.h}}
\subsubsection{\setlength{\rightskip}{0pt plus 5cm}\#define RR~1}\label{os_8h_63979cf6054f403eab1d354e6dcc4ce9}


A scheduling level: first-come-first-served cooperative tasks \begin{Desc}
\item[See also:]{\bf RR TASKS}{\rm (p.\,\pageref{index_rr})}, {\bf Task\_\-Create()}{\rm (p.\,\pageref{os_8c_d26f7e7c3185a703e89e97547747b03e})}. \end{Desc}
\index{os.h@{os.h}!SYSTEM@{SYSTEM}}
\index{SYSTEM@{SYSTEM}!os.h@{os.h}}
\subsubsection{\setlength{\rightskip}{0pt plus 5cm}\#define SYSTEM~3}\label{os_8h_21b97df85e65556468b28a576311271c}


a scheduling level: system tasks with first-come-first-served policy \begin{Desc}
\item[See also:]{\bf SYSTEM TASKS}{\rm (p.\,\pageref{index_system})}, {\bf Task\_\-Create()}{\rm (p.\,\pageref{os_8c_d26f7e7c3185a703e89e97547747b03e})}. \end{Desc}
\index{os.h@{os.h}!TICK@{TICK}}
\index{TICK@{TICK}!os.h@{os.h}}
\subsubsection{\setlength{\rightskip}{0pt plus 5cm}\#define TICK~5}\label{os_8h_7d5ee1cc1e801872efa1ea7486189019}


milliseconds, or something close to this value \begin{Desc}
\item[See also:]{\bf PERIODIC TASKS}{\rm (p.\,\pageref{index_periodic})}. \end{Desc}
\index{os.h@{os.h}!WORKSPACE@{WORKSPACE}}
\index{WORKSPACE@{WORKSPACE}!os.h@{os.h}}
\subsubsection{\setlength{\rightskip}{0pt plus 5cm}\#define WORKSPACE~512}\label{os_8h_420f71a6e3a201757631a92a1fb38529}


workspace size of each process in bytes 

\subsection{Typedef Documentation}
\index{os.h@{os.h}!EVENT@{EVENT}}
\index{EVENT@{EVENT}!os.h@{os.h}}
\subsubsection{\setlength{\rightskip}{0pt plus 5cm}typedef struct event {\bf EVENT}}\label{os_8h_a280985b6f47e1541bbf5012bf62c450}


An Event descriptor \begin{Desc}
\item[See also:]{\bf Event\_\-Init()}{\rm (p.\,\pageref{os_8c_852c4f8ed9198c7c996cc5e9ce7f1f2b})}. \end{Desc}


\subsection{Function Documentation}
\index{os.h@{os.h}!Broadcast_And_Next@{Broadcast\_\-And\_\-Next}}
\index{Broadcast_And_Next@{Broadcast\_\-And\_\-Next}!os.h@{os.h}}
\subsubsection{\setlength{\rightskip}{0pt plus 5cm}void Broadcast\_\-And\_\-Next ({\bf EVENT} $\ast$ {\em e})}\label{os_8h_8530ed5f351b8835f3b060407bd68625}


Resume {\bf ALL} waiting tasks on {\em e\/} and at the same time relinquish the processor. 

\begin{Desc}
\item[Parameters:]
\begin{description}
\item[{\em e}]an Event descriptor\end{description}
\end{Desc}
This is equivalent to \char`\"{}Event\_\-Broadcast( e ); Task\_\-Next()\char`\"{} in concept. \begin{Desc}
\item[See also:]{\bf Event\_\-Broadcast()}{\rm (p.\,\pageref{os_8c_93f9686857f1f8cc4fe1585df3de4874})}, {\bf Task\_\-Next()}{\rm (p.\,\pageref{os_8c_5a9b72d0dadaea32fec8d4ff1c0eafa4})} \end{Desc}
\index{os.h@{os.h}!Event_Broadcast@{Event\_\-Broadcast}}
\index{Event_Broadcast@{Event\_\-Broadcast}!os.h@{os.h}}
\subsubsection{\setlength{\rightskip}{0pt plus 5cm}void Event\_\-Broadcast ({\bf EVENT} $\ast$ {\em e})}\label{os_8h_93f9686857f1f8cc4fe1585df3de4874}


Resume {\bf ALL} waiting tasks on {\em e\/}. It is a {\em no-op\/} if there is no waiting process. 

\begin{Desc}
\item[Parameters:]
\begin{description}
\item[{\em e}]an Event descriptor\end{description}
\end{Desc}
\begin{Desc}
\item[See also:]{\bf Event\_\-Wait()}{\rm (p.\,\pageref{os_8c_e279428ffa0e59261a01899cea931503})} \end{Desc}
\index{os.h@{os.h}!Event_Init@{Event\_\-Init}}
\index{Event_Init@{Event\_\-Init}!os.h@{os.h}}
\subsubsection{\setlength{\rightskip}{0pt plus 5cm}{\bf EVENT}$\ast$ Event\_\-Init (void)}\label{os_8h_852c4f8ed9198c7c996cc5e9ce7f1f2b}


Initialize a new, non-NULL Event descriptor. 

\begin{Desc}
\item[Returns:]a non-NULL Event descriptor if successful; NULL otherwise. \end{Desc}
\index{os.h@{os.h}!Event_Signal@{Event\_\-Signal}}
\index{Event_Signal@{Event\_\-Signal}!os.h@{os.h}}
\subsubsection{\setlength{\rightskip}{0pt plus 5cm}void Event\_\-Signal ({\bf EVENT} $\ast$ {\em e})}\label{os_8h_39210e081be158dab105c68cb85585cd}


Resume a {\bf single} waiting task on {\em e\/}. It is a {\em no-op\/} if there is no waiting process. 

\begin{Desc}
\item[Parameters:]
\begin{description}
\item[{\em e}]an Event descriptor\end{description}
\end{Desc}
\begin{Desc}
\item[See also:]{\bf Event\_\-Wait()}{\rm (p.\,\pageref{os_8c_e279428ffa0e59261a01899cea931503})} \end{Desc}
\index{os.h@{os.h}!Event_Wait@{Event\_\-Wait}}
\index{Event_Wait@{Event\_\-Wait}!os.h@{os.h}}
\subsubsection{\setlength{\rightskip}{0pt plus 5cm}void Event\_\-Wait ({\bf EVENT} $\ast$ {\em e})}\label{os_8h_e279428ffa0e59261a01899cea931503}


Wait for the next occurrence of a signal on {\em e\/}. The calling process always blocks. 

\begin{Desc}
\item[Parameters:]
\begin{description}
\item[{\em e}]an Event descriptor \end{description}
\end{Desc}
\index{os.h@{os.h}!OS_Abort@{OS\_\-Abort}}
\index{OS_Abort@{OS\_\-Abort}!os.h@{os.h}}
\subsubsection{\setlength{\rightskip}{0pt plus 5cm}void OS\_\-Abort (void)}\label{os_8h_b5747390a8be675282cb93e5198bd085}


Abort the execution of this RTOS due to an unrecoverable erorr. 

Abort the execution of this RTOS due to an unrecoverable erorr. \begin{Desc}
\item[See also:]{\bf GLOBAL ASSUMPTIONS}{\rm (p.\,\pageref{index_assumptions})}. \end{Desc}
\index{os.h@{os.h}!OS_Init@{OS\_\-Init}}
\index{OS_Init@{OS\_\-Init}!os.h@{os.h}}
\subsubsection{\setlength{\rightskip}{0pt plus 5cm}void OS\_\-Init (void)}\label{os_8h_cb6df8f47f418aad9c9a9e045d7d1e6d}


Setup the RTOS and create {\bf main()}{\rm (p.\,\pageref{_assignment2_8c_840291bc02cba5474a4cb46a9b9566fe})} as the first SYSTEM level task. 

Point of entry from the C runtime crt0.S. \index{os.h@{os.h}!Signal_And_Next@{Signal\_\-And\_\-Next}}
\index{Signal_And_Next@{Signal\_\-And\_\-Next}!os.h@{os.h}}
\subsubsection{\setlength{\rightskip}{0pt plus 5cm}void Signal\_\-And\_\-Next ({\bf EVENT} $\ast$ {\em e})}\label{os_8h_899044f5840427560c88d0d6e8944629}


Resume a waiting task on {\em e\/} and at the same time relinquish the processor. 

\begin{Desc}
\item[Parameters:]
\begin{description}
\item[{\em e}]an Event descriptor\end{description}
\end{Desc}
This is equivalent to \char`\"{}Event\_\-Signal( e ); Task\_\-Next()\char`\"{} in concept. The fundamental difference is that these two operations are performed as an indivisible unit. So conceptually, the calling task resumes another waiting task and gives up its share of the processor simultaneously. \begin{Desc}
\item[See also:]{\bf Event\_\-Signal()}{\rm (p.\,\pageref{os_8c_39210e081be158dab105c68cb85585cd})}, {\bf Task\_\-Next()}{\rm (p.\,\pageref{os_8c_5a9b72d0dadaea32fec8d4ff1c0eafa4})} \end{Desc}
\index{os.h@{os.h}!Task_Create@{Task\_\-Create}}
\index{Task_Create@{Task\_\-Create}!os.h@{os.h}}
\subsubsection{\setlength{\rightskip}{0pt plus 5cm}int Task\_\-Create (void($\ast$)(void) {\em f}, int {\em arg}, unsigned int {\em level}, unsigned int {\em name})}\label{os_8h_d26f7e7c3185a703e89e97547747b03e}


\begin{Desc}
\item[Parameters:]
\begin{description}
\item[{\em f}]a parameterless function to be created as a process instance \item[{\em arg}]an integer argument to be assigned to this process instanace \item[{\em level}]assigned scheduling level: SYSTEM, PERIODIC or RR \item[{\em name}]assigned PERIODIC process name \end{description}
\end{Desc}
\begin{Desc}
\item[Returns:]0 if not successful; otherwise non-zero. \end{Desc}
\begin{Desc}
\item[See also:]{\bf Task\_\-Get\-Arg()}{\rm (p.\,\pageref{os_8c_dc48a5ac983c4656508f39c0ee65283f})}, {\bf PPP}{\rm (p.\,\pageref{os_8h_9139cb65cf60e47afed151765972c100})}[].\end{Desc}
A new process is created to execute the parameterless function {\em f\/} with an initial parameter {\em arg\/}, which is retrieved by a call to {\bf Task\_\-Get\-Arg()}{\rm (p.\,\pageref{os_8c_dc48a5ac983c4656508f39c0ee65283f})}. If a new process cannot be created, 0 is returned; otherwise, it returns non-zero. The created process will belong to its scheduling {\em level\/}. If the process is PERIODIC, then its {\em name\/} is a user-specified name to be used in the PPP[] array. Otherwise, {\em name\/} is ignored. \begin{Desc}
\item[See also:]{\bf SCHEDULING POLICY}{\rm (p.\,\pageref{index_policy})} \end{Desc}
\index{os.h@{os.h}!Task_GetArg@{Task\_\-GetArg}}
\index{Task_GetArg@{Task\_\-GetArg}!os.h@{os.h}}
\subsubsection{\setlength{\rightskip}{0pt plus 5cm}int Task\_\-Get\-Arg (void)}\label{os_8h_dc48a5ac983c4656508f39c0ee65283f}


Retrieve the assigned parameter. 

Retrieve the assigned parameter. \begin{Desc}
\item[See also:]{\bf Task\_\-Create()}{\rm (p.\,\pageref{os_8c_d26f7e7c3185a703e89e97547747b03e})}. \end{Desc}
\index{os.h@{os.h}!Task_Next@{Task\_\-Next}}
\index{Task_Next@{Task\_\-Next}!os.h@{os.h}}
\subsubsection{\setlength{\rightskip}{0pt plus 5cm}void Task\_\-Next (void)}\label{os_8h_df2f899160bcb12d32f2cbce83470ea5}


The calling task gives up its share of the processor voluntarily. 

Voluntarily relinquish the processor. \index{os.h@{os.h}!Task_Terminate@{Task\_\-Terminate}}
\index{Task_Terminate@{Task\_\-Terminate}!os.h@{os.h}}
\subsubsection{\setlength{\rightskip}{0pt plus 5cm}void Task\_\-Terminate (void)}\label{os_8h_eb1fcf3f13ebc649c9882f07de88cf74}


The calling task terminates itself. 

Terminate the calling process

When a process returns, i.e., it executes its last instruction in the associated function/code, it is automatically terminated. 

\subsection{Variable Documentation}
\index{os.h@{os.h}!PPP@{PPP}}
\index{PPP@{PPP}!os.h@{os.h}}
\subsubsection{\setlength{\rightskip}{0pt plus 5cm}const unsigned char {\bf PPP}[$\,$]}\label{os_8h_9139cb65cf60e47afed151765972c100}


PPP and PT defined in user application. \index{os.h@{os.h}!PT@{PT}}
\index{PT@{PT}!os.h@{os.h}}
\subsubsection{\setlength{\rightskip}{0pt plus 5cm}const unsigned int {\bf PT}}\label{os_8h_6370d00b2f49ebb4a69edf643003ada2}


PPP and PT defined in user application. 